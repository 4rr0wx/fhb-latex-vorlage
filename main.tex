\documentclass[12pt,a4paper]{article}
\usepackage[utf8]{inputenc}
%ENtfernt rote ränder von Hyperlinks
\usepackage[hidelinks]{hyperref}
%Wird nicht mehr benötigt damit Umlaute im PDF richtig angezeigt werden
%\usepackage[T1]{fontenc}	

%Wird fürs Anzeigen von Grafiken benötigt
\usepackage{graphicx}

%Wird benötigt damit Tabellenbeschriftung auf Deutsch
\usepackage[ngerman]{babel} 

% verwendet für subfigure Umgebungen
\usepackage{subcaption}

% verwendet für todo notes (\todo)
\setlength {\marginparwidth }{2cm}
\usepackage[colorinlistoftodos]{todonotes}

%Package für Listing (code listing - Umgebung "lstlisting")
\usepackage{listings}

%Packages für Quellenverzeichnis
% Package für \url Befehl
\usepackage{url}
% Package-Abhängigkeit für \appto Befehl (wird für den nächsten Kommando (UrlBreaks) benötigt)
\usepackage{etoolbox}

%wird für autoref benötigt, andererseits erstellt es im pdf klickbare Referenzen
\usepackage{hyperref}
\usepackage{amsmath}
%wird für cref benötigt
\usepackage[ngerman]{cleveref}

\usepackage{relsize}
\usepackage[ngerman]{babel}
\usepackage{gensymb} 
\usepackage{textcomp}

\usepackage[sfdefault]{arimo}
% Font encoding 
\usepackage[T1]{fontenc}  
% This package allows the user to specify the input encoding 
\usepackage[utf8]{inputenc}
\usepackage{setspace}

%eventuell benötigte Pakete

 % Das Paket wrapfig ermöglicht es von Schrift umflossene Bilder und Tabellen
% \usepackage{wrapfig}

% ermöglicht das anpassen der list Umgebungen (itemize, enumerate, description)
% \usepackage{enumitem}

% bessere Möglichkeiten in die Platzierung von Abbildungen, etc. einzugreifen
% \usepackage{float}

% Abstand zwischen zwei Absätzen kontrollieren 
% \usepackage{parskip}

% konfiguriert Parameter des hyperref packages, so dass Linkfarben angepasst werden - muss die parameterlose Variante des Befehls (\usepackage{hyperref}) ersetzen
% \usepackage[colorlinks=true, urlcolor=blue, linkcolor=black]{hyperref}

\usepackage[acronym]{glossaries}
\usepackage{blindtext}
\usepackage{tocloft}
\usepackage{lastpage}

%\do\? == erlaubtes Zeichen, an dem ein Zeilenumbruch innerhalb der URL stattfinden darf (auch mehrere \do\? erlaubt, z.B. \do\a\do\b\do\c...)
\appto\UrlBreaks{\do\4}
\makeglossaries


%Metadaten
% werden verwendet, wenn z.B. für das Titelblatt/Deckblatt der Befehl \maketitle verwendet wird
\title{FH-Burgenland Vorlage V2}
\author{Martin Scheifinger}
\date{\today} %{11.10.2020}


%Commandodefinition
% Defines a new command for the horizontal lines, change thickness here
% \newcommand{\HRule}{\rule{\linewidth}{0.5mm}} 

%Seitenränder und Druckbereich sowie Kopf- und Fußzeile definieren
\setlength\paperwidth{20.999cm}
\setlength\paperheight{29.699cm}
\setlength\voffset{-2cm}
\setlength\hoffset{-1cm}
\setlength\headheight{3.5cm}
\setlength\headsep{0.5cm}
\setlength\footskip{1.131cm}
\setlength\textheight{22cm}
\setlength\textwidth{17cm}
%\setlength\marginparwidth{2cm}

%%%%%%%%%%%%%%%%%%%%%%%%%%%%%%%%%%%%%%%%%%%% Kopfzeile und Fußzeile definieren

\usepackage{fancyhdr}
\pagestyle{fancy} %eigener Seitenstil
\fancyhf{} %alle Kopf und Fußzeilenfelder bereinigen
\fancyhead[L]{\KlassenNr - \FachKuerzel}
\fancyhead[R]{\includegraphics[height=1.7cm]{FH-Burgenland_Logos/Hochschule_Bgld_Logo_RGB.png}}


\let\oldheadrule\headrule% Copy \headrule into \oldheadrule

%uncommend below to make header line türkis
%\renewcommand{\headrule}{\color{teal}\oldheadrule}% Add colour to \headrule

\renewcommand{\headrulewidth}{0.5pt}%obere Trennlinie
%\renewcommand{\footrulewidth}{0.0pt} %untere Trennlinie definieren
\fancyfoot[L]{\SchuelerName}
\fancyfoot[R]{Seite \thepage \hspace{1pt} von \pageref{LastPage}}
\pagenumbering{arabic}

%%%%%%%%%%%%%%%%%%%%%%%%% Inhaltsverzeichnis Formatierung definieren

\cftsetindents{section}{0em}{3em}
\cftsetindents{subsection}{0em}{3em}
\cftsetindents{subsubsection}{0em}{3em}

\setcounter{tocdepth}{4} %Damit Inhaltsverzeichnis bis zur 4 Ebene
\renewcommand\thesection{\arabic{section}} %Damit die Nummerierung mit 1 beginnt, vgl.: \roman{section}

\setcounter{secnumdepth}{3} %Damit Nummerierung bis zur Ebene 3 geht






%%%%%%%%%%%%%%%%%%%%%%%%%%%%%%%%%%%%%%%%%%%%%%%%%%%%%%%%%%%%%%%%%%%%%%%%%%%%%%%%
%Start vom eigentlichen Dokument / Schriftstück 
\begin{document}
%Deckblatt
\include{Dokument/Deckblatt}

\newpage
%%%%%%%%%%%%%%%%%%%%%%%%%%%%%%%%%%%% Ende Deckblatt
%%%%%%%%%%%%%%%%%%%%%%%%%%%%%%%%%%%% Aktivieren der verschiedenen Verzeichnisse
% wobei Quell- und Literaturverzeichnis sich am Ende des Dokumentes befinden.

\tableofcontents\thispagestyle{fancy}


\pagenumbering{arabic}
%\setcounter{page}{1}

%%%%%%%%%%%%%%%%%%%%%%%%%%%%%%% Ende Verzeichnis
%_______________________________________________________________________________________________________
%   ###########################################################
%   #           Metadaten ändern nicht vergessen              #
%   ###########################################################
% Dokument: main
%   Zeile: 36    | Header: Fach Hinzufügen ()
% Dokument: Deckblatt:
%   Zeile: 12    | Titel des Dokuments     ()
%   Zeile: 15&17 | Überschriften           ()
%   Zeile: 24    | Vortragende Person      ()
%_______________________________________________________________________________________________________
%--------------Wichtige Metadaten für jedes Dokument ------------------
\def\FachKuerzel{SIBA}
\def\KlassenNr{BITI - 6V} %z.B. BITI - 5V
\def\Titel{Notfallkonzept Gruppenarbeit}
\def\Ueberschrifta{Notfallhandbuch}
\def\Ueberschriftb{Operative Umsetzungen} %Falls nicht benötigt in File Dekblatt Zeile 19 auskommentieren
\def\Lehrer{Marcus Zajer}


%--------------- Name des Schülers ------------------------
\def\SchuelerName{Sieber, Nährer, Böhm, Guttman, Scheifinger}
\def\SchuelerPKZ{,,,,2210640020}
\clearpage



%%%%%%%%%%%%%%%%%%%%%%%%%%%%%%%%%%%%% Anfang Dokument %%%%%%%%%%%%%%%%%%%%%%%%%%%%%%%%%%%%%%%%%%%%%%

% ----------------------  Kapitel 1 -----------------------
\newpage
\section{Kapitel 1}
% the Asterix (*) indicates that this section will be added to the table of contents but no number will be present beside it.
\section{Einleitung}


\section{Fazit \& Ausblick}



%%%%%%%%%%%%%%%%%%%%%%%%%%%%%%%%%%%%%%%%%%%%%%%%%%%%%%%%%%%%%%%%%%%%%%%%%%%%%%%%%%%%%%%%%%%%%%%%%%%%
\newpage
\section{Anhang}
%Hier einkommentieren wenn man Abbildungsverzeichnis oder Tabellenverzeichnis macht
%\listoffigures\thispagestyle{fancy}
%\clearpage
%\listoftables\thispagestyle{fancy}
%\clearpage
% -----------------  ACRONYMS  -------------------


\printglossary[type=\acronymtype,nonumberlist,nopostdot]



\end{document}
